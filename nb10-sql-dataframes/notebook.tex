
% Default to the notebook output style

    


% Inherit from the specified cell style.




    
\documentclass[11pt]{article}

    
    
    \usepackage[T1]{fontenc}
    % Nicer default font (+ math font) than Computer Modern for most use cases
    \usepackage{mathpazo}

    % Basic figure setup, for now with no caption control since it's done
    % automatically by Pandoc (which extracts ![](path) syntax from Markdown).
    \usepackage{graphicx}
    % We will generate all images so they have a width \maxwidth. This means
    % that they will get their normal width if they fit onto the page, but
    % are scaled down if they would overflow the margins.
    \makeatletter
    \def\maxwidth{\ifdim\Gin@nat@width>\linewidth\linewidth
    \else\Gin@nat@width\fi}
    \makeatother
    \let\Oldincludegraphics\includegraphics
    % Set max figure width to be 80% of text width, for now hardcoded.
    \renewcommand{\includegraphics}[1]{\Oldincludegraphics[width=.8\maxwidth]{#1}}
    % Ensure that by default, figures have no caption (until we provide a
    % proper Figure object with a Caption API and a way to capture that
    % in the conversion process - todo).
    \usepackage{caption}
    \DeclareCaptionLabelFormat{nolabel}{}
    \captionsetup{labelformat=nolabel}

    \usepackage{adjustbox} % Used to constrain images to a maximum size 
    \usepackage{xcolor} % Allow colors to be defined
    \usepackage{enumerate} % Needed for markdown enumerations to work
    \usepackage{geometry} % Used to adjust the document margins
    \usepackage{amsmath} % Equations
    \usepackage{amssymb} % Equations
    \usepackage{textcomp} % defines textquotesingle
    % Hack from http://tex.stackexchange.com/a/47451/13684:
    \AtBeginDocument{%
        \def\PYZsq{\textquotesingle}% Upright quotes in Pygmentized code
    }
    \usepackage{upquote} % Upright quotes for verbatim code
    \usepackage{eurosym} % defines \euro
    \usepackage[mathletters]{ucs} % Extended unicode (utf-8) support
    \usepackage[utf8x]{inputenc} % Allow utf-8 characters in the tex document
    \usepackage{fancyvrb} % verbatim replacement that allows latex
    \usepackage{grffile} % extends the file name processing of package graphics 
                         % to support a larger range 
    % The hyperref package gives us a pdf with properly built
    % internal navigation ('pdf bookmarks' for the table of contents,
    % internal cross-reference links, web links for URLs, etc.)
    \usepackage{hyperref}
    \usepackage{longtable} % longtable support required by pandoc >1.10
    \usepackage{booktabs}  % table support for pandoc > 1.12.2
    \usepackage[inline]{enumitem} % IRkernel/repr support (it uses the enumerate* environment)
    \usepackage[normalem]{ulem} % ulem is needed to support strikethroughs (\sout)
                                % normalem makes italics be italics, not underlines
    

    
    
    % Colors for the hyperref package
    \definecolor{urlcolor}{rgb}{0,.145,.698}
    \definecolor{linkcolor}{rgb}{.71,0.21,0.01}
    \definecolor{citecolor}{rgb}{.12,.54,.11}

    % ANSI colors
    \definecolor{ansi-black}{HTML}{3E424D}
    \definecolor{ansi-black-intense}{HTML}{282C36}
    \definecolor{ansi-red}{HTML}{E75C58}
    \definecolor{ansi-red-intense}{HTML}{B22B31}
    \definecolor{ansi-green}{HTML}{00A250}
    \definecolor{ansi-green-intense}{HTML}{007427}
    \definecolor{ansi-yellow}{HTML}{DDB62B}
    \definecolor{ansi-yellow-intense}{HTML}{B27D12}
    \definecolor{ansi-blue}{HTML}{208FFB}
    \definecolor{ansi-blue-intense}{HTML}{0065CA}
    \definecolor{ansi-magenta}{HTML}{D160C4}
    \definecolor{ansi-magenta-intense}{HTML}{A03196}
    \definecolor{ansi-cyan}{HTML}{60C6C8}
    \definecolor{ansi-cyan-intense}{HTML}{258F8F}
    \definecolor{ansi-white}{HTML}{C5C1B4}
    \definecolor{ansi-white-intense}{HTML}{A1A6B2}

    % commands and environments needed by pandoc snippets
    % extracted from the output of `pandoc -s`
    \providecommand{\tightlist}{%
      \setlength{\itemsep}{0pt}\setlength{\parskip}{0pt}}
    \DefineVerbatimEnvironment{Highlighting}{Verbatim}{commandchars=\\\{\}}
    % Add ',fontsize=\small' for more characters per line
    \newenvironment{Shaded}{}{}
    \newcommand{\KeywordTok}[1]{\textcolor[rgb]{0.00,0.44,0.13}{\textbf{{#1}}}}
    \newcommand{\DataTypeTok}[1]{\textcolor[rgb]{0.56,0.13,0.00}{{#1}}}
    \newcommand{\DecValTok}[1]{\textcolor[rgb]{0.25,0.63,0.44}{{#1}}}
    \newcommand{\BaseNTok}[1]{\textcolor[rgb]{0.25,0.63,0.44}{{#1}}}
    \newcommand{\FloatTok}[1]{\textcolor[rgb]{0.25,0.63,0.44}{{#1}}}
    \newcommand{\CharTok}[1]{\textcolor[rgb]{0.25,0.44,0.63}{{#1}}}
    \newcommand{\StringTok}[1]{\textcolor[rgb]{0.25,0.44,0.63}{{#1}}}
    \newcommand{\CommentTok}[1]{\textcolor[rgb]{0.38,0.63,0.69}{\textit{{#1}}}}
    \newcommand{\OtherTok}[1]{\textcolor[rgb]{0.00,0.44,0.13}{{#1}}}
    \newcommand{\AlertTok}[1]{\textcolor[rgb]{1.00,0.00,0.00}{\textbf{{#1}}}}
    \newcommand{\FunctionTok}[1]{\textcolor[rgb]{0.02,0.16,0.49}{{#1}}}
    \newcommand{\RegionMarkerTok}[1]{{#1}}
    \newcommand{\ErrorTok}[1]{\textcolor[rgb]{1.00,0.00,0.00}{\textbf{{#1}}}}
    \newcommand{\NormalTok}[1]{{#1}}
    
    % Additional commands for more recent versions of Pandoc
    \newcommand{\ConstantTok}[1]{\textcolor[rgb]{0.53,0.00,0.00}{{#1}}}
    \newcommand{\SpecialCharTok}[1]{\textcolor[rgb]{0.25,0.44,0.63}{{#1}}}
    \newcommand{\VerbatimStringTok}[1]{\textcolor[rgb]{0.25,0.44,0.63}{{#1}}}
    \newcommand{\SpecialStringTok}[1]{\textcolor[rgb]{0.73,0.40,0.53}{{#1}}}
    \newcommand{\ImportTok}[1]{{#1}}
    \newcommand{\DocumentationTok}[1]{\textcolor[rgb]{0.73,0.13,0.13}{\textit{{#1}}}}
    \newcommand{\AnnotationTok}[1]{\textcolor[rgb]{0.38,0.63,0.69}{\textbf{\textit{{#1}}}}}
    \newcommand{\CommentVarTok}[1]{\textcolor[rgb]{0.38,0.63,0.69}{\textbf{\textit{{#1}}}}}
    \newcommand{\VariableTok}[1]{\textcolor[rgb]{0.10,0.09,0.49}{{#1}}}
    \newcommand{\ControlFlowTok}[1]{\textcolor[rgb]{0.00,0.44,0.13}{\textbf{{#1}}}}
    \newcommand{\OperatorTok}[1]{\textcolor[rgb]{0.40,0.40,0.40}{{#1}}}
    \newcommand{\BuiltInTok}[1]{{#1}}
    \newcommand{\ExtensionTok}[1]{{#1}}
    \newcommand{\PreprocessorTok}[1]{\textcolor[rgb]{0.74,0.48,0.00}{{#1}}}
    \newcommand{\AttributeTok}[1]{\textcolor[rgb]{0.49,0.56,0.16}{{#1}}}
    \newcommand{\InformationTok}[1]{\textcolor[rgb]{0.38,0.63,0.69}{\textbf{\textit{{#1}}}}}
    \newcommand{\WarningTok}[1]{\textcolor[rgb]{0.38,0.63,0.69}{\textbf{\textit{{#1}}}}}
    
    
    % Define a nice break command that doesn't care if a line doesn't already
    % exist.
    \def\br{\hspace*{\fill} \\* }
    % Math Jax compatability definitions
    \def\gt{>}
    \def\lt{<}
    % Document parameters
    \title{nb10-sql-dataframes}
    
    
    

    % Pygments definitions
    
\makeatletter
\def\PY@reset{\let\PY@it=\relax \let\PY@bf=\relax%
    \let\PY@ul=\relax \let\PY@tc=\relax%
    \let\PY@bc=\relax \let\PY@ff=\relax}
\def\PY@tok#1{\csname PY@tok@#1\endcsname}
\def\PY@toks#1+{\ifx\relax#1\empty\else%
    \PY@tok{#1}\expandafter\PY@toks\fi}
\def\PY@do#1{\PY@bc{\PY@tc{\PY@ul{%
    \PY@it{\PY@bf{\PY@ff{#1}}}}}}}
\def\PY#1#2{\PY@reset\PY@toks#1+\relax+\PY@do{#2}}

\expandafter\def\csname PY@tok@w\endcsname{\def\PY@tc##1{\textcolor[rgb]{0.73,0.73,0.73}{##1}}}
\expandafter\def\csname PY@tok@c\endcsname{\let\PY@it=\textit\def\PY@tc##1{\textcolor[rgb]{0.25,0.50,0.50}{##1}}}
\expandafter\def\csname PY@tok@cp\endcsname{\def\PY@tc##1{\textcolor[rgb]{0.74,0.48,0.00}{##1}}}
\expandafter\def\csname PY@tok@k\endcsname{\let\PY@bf=\textbf\def\PY@tc##1{\textcolor[rgb]{0.00,0.50,0.00}{##1}}}
\expandafter\def\csname PY@tok@kp\endcsname{\def\PY@tc##1{\textcolor[rgb]{0.00,0.50,0.00}{##1}}}
\expandafter\def\csname PY@tok@kt\endcsname{\def\PY@tc##1{\textcolor[rgb]{0.69,0.00,0.25}{##1}}}
\expandafter\def\csname PY@tok@o\endcsname{\def\PY@tc##1{\textcolor[rgb]{0.40,0.40,0.40}{##1}}}
\expandafter\def\csname PY@tok@ow\endcsname{\let\PY@bf=\textbf\def\PY@tc##1{\textcolor[rgb]{0.67,0.13,1.00}{##1}}}
\expandafter\def\csname PY@tok@nb\endcsname{\def\PY@tc##1{\textcolor[rgb]{0.00,0.50,0.00}{##1}}}
\expandafter\def\csname PY@tok@nf\endcsname{\def\PY@tc##1{\textcolor[rgb]{0.00,0.00,1.00}{##1}}}
\expandafter\def\csname PY@tok@nc\endcsname{\let\PY@bf=\textbf\def\PY@tc##1{\textcolor[rgb]{0.00,0.00,1.00}{##1}}}
\expandafter\def\csname PY@tok@nn\endcsname{\let\PY@bf=\textbf\def\PY@tc##1{\textcolor[rgb]{0.00,0.00,1.00}{##1}}}
\expandafter\def\csname PY@tok@ne\endcsname{\let\PY@bf=\textbf\def\PY@tc##1{\textcolor[rgb]{0.82,0.25,0.23}{##1}}}
\expandafter\def\csname PY@tok@nv\endcsname{\def\PY@tc##1{\textcolor[rgb]{0.10,0.09,0.49}{##1}}}
\expandafter\def\csname PY@tok@no\endcsname{\def\PY@tc##1{\textcolor[rgb]{0.53,0.00,0.00}{##1}}}
\expandafter\def\csname PY@tok@nl\endcsname{\def\PY@tc##1{\textcolor[rgb]{0.63,0.63,0.00}{##1}}}
\expandafter\def\csname PY@tok@ni\endcsname{\let\PY@bf=\textbf\def\PY@tc##1{\textcolor[rgb]{0.60,0.60,0.60}{##1}}}
\expandafter\def\csname PY@tok@na\endcsname{\def\PY@tc##1{\textcolor[rgb]{0.49,0.56,0.16}{##1}}}
\expandafter\def\csname PY@tok@nt\endcsname{\let\PY@bf=\textbf\def\PY@tc##1{\textcolor[rgb]{0.00,0.50,0.00}{##1}}}
\expandafter\def\csname PY@tok@nd\endcsname{\def\PY@tc##1{\textcolor[rgb]{0.67,0.13,1.00}{##1}}}
\expandafter\def\csname PY@tok@s\endcsname{\def\PY@tc##1{\textcolor[rgb]{0.73,0.13,0.13}{##1}}}
\expandafter\def\csname PY@tok@sd\endcsname{\let\PY@it=\textit\def\PY@tc##1{\textcolor[rgb]{0.73,0.13,0.13}{##1}}}
\expandafter\def\csname PY@tok@si\endcsname{\let\PY@bf=\textbf\def\PY@tc##1{\textcolor[rgb]{0.73,0.40,0.53}{##1}}}
\expandafter\def\csname PY@tok@se\endcsname{\let\PY@bf=\textbf\def\PY@tc##1{\textcolor[rgb]{0.73,0.40,0.13}{##1}}}
\expandafter\def\csname PY@tok@sr\endcsname{\def\PY@tc##1{\textcolor[rgb]{0.73,0.40,0.53}{##1}}}
\expandafter\def\csname PY@tok@ss\endcsname{\def\PY@tc##1{\textcolor[rgb]{0.10,0.09,0.49}{##1}}}
\expandafter\def\csname PY@tok@sx\endcsname{\def\PY@tc##1{\textcolor[rgb]{0.00,0.50,0.00}{##1}}}
\expandafter\def\csname PY@tok@m\endcsname{\def\PY@tc##1{\textcolor[rgb]{0.40,0.40,0.40}{##1}}}
\expandafter\def\csname PY@tok@gh\endcsname{\let\PY@bf=\textbf\def\PY@tc##1{\textcolor[rgb]{0.00,0.00,0.50}{##1}}}
\expandafter\def\csname PY@tok@gu\endcsname{\let\PY@bf=\textbf\def\PY@tc##1{\textcolor[rgb]{0.50,0.00,0.50}{##1}}}
\expandafter\def\csname PY@tok@gd\endcsname{\def\PY@tc##1{\textcolor[rgb]{0.63,0.00,0.00}{##1}}}
\expandafter\def\csname PY@tok@gi\endcsname{\def\PY@tc##1{\textcolor[rgb]{0.00,0.63,0.00}{##1}}}
\expandafter\def\csname PY@tok@gr\endcsname{\def\PY@tc##1{\textcolor[rgb]{1.00,0.00,0.00}{##1}}}
\expandafter\def\csname PY@tok@ge\endcsname{\let\PY@it=\textit}
\expandafter\def\csname PY@tok@gs\endcsname{\let\PY@bf=\textbf}
\expandafter\def\csname PY@tok@gp\endcsname{\let\PY@bf=\textbf\def\PY@tc##1{\textcolor[rgb]{0.00,0.00,0.50}{##1}}}
\expandafter\def\csname PY@tok@go\endcsname{\def\PY@tc##1{\textcolor[rgb]{0.53,0.53,0.53}{##1}}}
\expandafter\def\csname PY@tok@gt\endcsname{\def\PY@tc##1{\textcolor[rgb]{0.00,0.27,0.87}{##1}}}
\expandafter\def\csname PY@tok@err\endcsname{\def\PY@bc##1{\setlength{\fboxsep}{0pt}\fcolorbox[rgb]{1.00,0.00,0.00}{1,1,1}{\strut ##1}}}
\expandafter\def\csname PY@tok@kc\endcsname{\let\PY@bf=\textbf\def\PY@tc##1{\textcolor[rgb]{0.00,0.50,0.00}{##1}}}
\expandafter\def\csname PY@tok@kd\endcsname{\let\PY@bf=\textbf\def\PY@tc##1{\textcolor[rgb]{0.00,0.50,0.00}{##1}}}
\expandafter\def\csname PY@tok@kn\endcsname{\let\PY@bf=\textbf\def\PY@tc##1{\textcolor[rgb]{0.00,0.50,0.00}{##1}}}
\expandafter\def\csname PY@tok@kr\endcsname{\let\PY@bf=\textbf\def\PY@tc##1{\textcolor[rgb]{0.00,0.50,0.00}{##1}}}
\expandafter\def\csname PY@tok@bp\endcsname{\def\PY@tc##1{\textcolor[rgb]{0.00,0.50,0.00}{##1}}}
\expandafter\def\csname PY@tok@fm\endcsname{\def\PY@tc##1{\textcolor[rgb]{0.00,0.00,1.00}{##1}}}
\expandafter\def\csname PY@tok@vc\endcsname{\def\PY@tc##1{\textcolor[rgb]{0.10,0.09,0.49}{##1}}}
\expandafter\def\csname PY@tok@vg\endcsname{\def\PY@tc##1{\textcolor[rgb]{0.10,0.09,0.49}{##1}}}
\expandafter\def\csname PY@tok@vi\endcsname{\def\PY@tc##1{\textcolor[rgb]{0.10,0.09,0.49}{##1}}}
\expandafter\def\csname PY@tok@vm\endcsname{\def\PY@tc##1{\textcolor[rgb]{0.10,0.09,0.49}{##1}}}
\expandafter\def\csname PY@tok@sa\endcsname{\def\PY@tc##1{\textcolor[rgb]{0.73,0.13,0.13}{##1}}}
\expandafter\def\csname PY@tok@sb\endcsname{\def\PY@tc##1{\textcolor[rgb]{0.73,0.13,0.13}{##1}}}
\expandafter\def\csname PY@tok@sc\endcsname{\def\PY@tc##1{\textcolor[rgb]{0.73,0.13,0.13}{##1}}}
\expandafter\def\csname PY@tok@dl\endcsname{\def\PY@tc##1{\textcolor[rgb]{0.73,0.13,0.13}{##1}}}
\expandafter\def\csname PY@tok@s2\endcsname{\def\PY@tc##1{\textcolor[rgb]{0.73,0.13,0.13}{##1}}}
\expandafter\def\csname PY@tok@sh\endcsname{\def\PY@tc##1{\textcolor[rgb]{0.73,0.13,0.13}{##1}}}
\expandafter\def\csname PY@tok@s1\endcsname{\def\PY@tc##1{\textcolor[rgb]{0.73,0.13,0.13}{##1}}}
\expandafter\def\csname PY@tok@mb\endcsname{\def\PY@tc##1{\textcolor[rgb]{0.40,0.40,0.40}{##1}}}
\expandafter\def\csname PY@tok@mf\endcsname{\def\PY@tc##1{\textcolor[rgb]{0.40,0.40,0.40}{##1}}}
\expandafter\def\csname PY@tok@mh\endcsname{\def\PY@tc##1{\textcolor[rgb]{0.40,0.40,0.40}{##1}}}
\expandafter\def\csname PY@tok@mi\endcsname{\def\PY@tc##1{\textcolor[rgb]{0.40,0.40,0.40}{##1}}}
\expandafter\def\csname PY@tok@il\endcsname{\def\PY@tc##1{\textcolor[rgb]{0.40,0.40,0.40}{##1}}}
\expandafter\def\csname PY@tok@mo\endcsname{\def\PY@tc##1{\textcolor[rgb]{0.40,0.40,0.40}{##1}}}
\expandafter\def\csname PY@tok@ch\endcsname{\let\PY@it=\textit\def\PY@tc##1{\textcolor[rgb]{0.25,0.50,0.50}{##1}}}
\expandafter\def\csname PY@tok@cm\endcsname{\let\PY@it=\textit\def\PY@tc##1{\textcolor[rgb]{0.25,0.50,0.50}{##1}}}
\expandafter\def\csname PY@tok@cpf\endcsname{\let\PY@it=\textit\def\PY@tc##1{\textcolor[rgb]{0.25,0.50,0.50}{##1}}}
\expandafter\def\csname PY@tok@c1\endcsname{\let\PY@it=\textit\def\PY@tc##1{\textcolor[rgb]{0.25,0.50,0.50}{##1}}}
\expandafter\def\csname PY@tok@cs\endcsname{\let\PY@it=\textit\def\PY@tc##1{\textcolor[rgb]{0.25,0.50,0.50}{##1}}}

\def\PYZbs{\char`\\}
\def\PYZus{\char`\_}
\def\PYZob{\char`\{}
\def\PYZcb{\char`\}}
\def\PYZca{\char`\^}
\def\PYZam{\char`\&}
\def\PYZlt{\char`\<}
\def\PYZgt{\char`\>}
\def\PYZsh{\char`\#}
\def\PYZpc{\char`\%}
\def\PYZdl{\char`\$}
\def\PYZhy{\char`\-}
\def\PYZsq{\char`\'}
\def\PYZdq{\char`\"}
\def\PYZti{\char`\~}
% for compatibility with earlier versions
\def\PYZat{@}
\def\PYZlb{[}
\def\PYZrb{]}
\makeatother


    % Exact colors from NB
    \definecolor{incolor}{rgb}{0.0, 0.0, 0.5}
    \definecolor{outcolor}{rgb}{0.545, 0.0, 0.0}



    
    % Prevent overflowing lines due to hard-to-break entities
    \sloppy 
    % Setup hyperref package
    \hypersetup{
      breaklinks=true,  % so long urls are correctly broken across lines
      colorlinks=true,
      urlcolor=urlcolor,
      linkcolor=linkcolor,
      citecolor=citecolor,
      }
    % Slightly bigger margins than the latex defaults
    
    \geometry{verbose,tmargin=1in,bmargin=1in,lmargin=1in,rmargin=1in}
    
    

    \begin{document}
    
    
    \maketitle
    
    

    
    \section{Spark SQL e Data Frames}\label{spark-sql-e-data-frames}

    \paragraph{\texorpdfstring{\href{https://github.com/jadianes/spark-py-notebooks}{Baseado
em "Introduction to Spark with Python, by Jose A.
Dianes"}}{Baseado em "Introduction to Spark with Python, by Jose A. Dianes"}}\label{baseado-em-introduction-to-spark-with-python-by-jose-a.-dianes}

    Este notebook apresentará os recursos do Spark para lidar com dados de
maneira estruturada. Basicamente, tudo gira em torno do conceito de
\emph{Data Frame} e usando \emph{SQL language} para consultá-los.
Veremos como a abstração do quadro de dados, muito popular em outros
ecossistemas de análise de dados (por exemplo, R e Python/Pandas), é
muito poderosa ao executar a análise exploratória de dados. De fato, é
muito fácil expressar consultas de dados quando usadas em conjunto com a
linguagem SQL. Além disso, o Spark distribui essa estrutura de dados
baseada em colunas de forma transparente, a fim de tornar o processo de
consulta o mais eficiente possível.

    \subsection{Fazendo download do
Dataset}\label{fazendo-download-do-dataset}

    Neste notebook, usaremos o conjunto de dados reduzido (10 por cento)
fornecido para a KDD Cup 1999, contendo quase meio milhão de interações
de rede. O arquivo é fornecido como um arquivo \emph{Gzip} que será
baixado localmente.

    \begin{Verbatim}[commandchars=\\\{\}]
{\color{incolor}In [{\color{incolor}1}]:} \PY{k+kn}{from} \PY{n+nn}{urllib}\PY{n+nn}{.}\PY{n+nn}{request} \PY{k}{import} \PY{n}{urlretrieve}
        
        \PY{n}{f} \PY{o}{=} \PY{n}{urlretrieve}\PY{p}{(}\PY{l+s+s2}{\PYZdq{}}\PY{l+s+s2}{http://kdd.ics.uci.edu/databases/kddcup99/kddcup.data\PYZus{}10\PYZus{}percent.gz}\PY{l+s+s2}{\PYZdq{}}\PY{p}{,} \PY{l+s+s2}{\PYZdq{}}\PY{l+s+s2}{kddcup.data\PYZus{}10\PYZus{}percent.gz}\PY{l+s+s2}{\PYZdq{}}\PY{p}{)}
\end{Verbatim}


    \subsection{Obtendo os dados e criando o
RDD}\label{obtendo-os-dados-e-criando-o-rdd}

    Como fizemos nos cadernos anteriores, usaremos o conjunto de dados
reduzido (10\%) fornecido para a
\href{http://kdd.ics.uci.edu/databases/kddcup99/kddcup99.html}{KDD Cup
1999}, contendo quase a metade milhões de interações nework. O arquivo é
fornecido como um arquivo Gzip que será baixado localmente.

    \begin{Verbatim}[commandchars=\\\{\}]
{\color{incolor}In [{\color{incolor}2}]:} \PY{n}{data\PYZus{}file} \PY{o}{=} \PY{l+s+s2}{\PYZdq{}}\PY{l+s+s2}{./kddcup.data\PYZus{}10\PYZus{}percent.gz}\PY{l+s+s2}{\PYZdq{}}
        \PY{n}{raw\PYZus{}data} \PY{o}{=} \PY{n}{sc}\PY{o}{.}\PY{n}{textFile}\PY{p}{(}\PY{n}{data\PYZus{}file}\PY{p}{)}\PY{o}{.}\PY{n}{cache}\PY{p}{(}\PY{p}{)}
\end{Verbatim}


    \subsection{Obtendo um quadro de
dados}\label{obtendo-um-quadro-de-dados}

    Um Spark \texttt{DataFrame} é uma coleção distribuída de dados
organizados em colunas nomeadas. É conceitualmente equivalente a uma
tabela em um banco de dados relacional ou a um quadro de dados em R ou
Pandas. Eles podem ser construídos a partir de uma ampla variedade de
fontes, como um RDD existente no nosso caso.

    O ponto de entrada em toda a funcionalidade do SQL no Spark é a classe
\texttt{SQLContext}. Para criar uma instância básica, tudo o que
precisamos é de uma referência ao \texttt{SparkContext}. Como estamos
executando o Spark no modo shell (usando pySpark), podemos usar o objeto
de contexto global \texttt{sc} para essa finalidade.

    \begin{Verbatim}[commandchars=\\\{\}]
{\color{incolor}In [{\color{incolor}3}]:} \PY{k+kn}{from} \PY{n+nn}{pyspark}\PY{n+nn}{.}\PY{n+nn}{sql} \PY{k}{import} \PY{n}{SQLContext}
        \PY{n}{sqlContext} \PY{o}{=} \PY{n}{SQLContext}\PY{p}{(}\PY{n}{sc}\PY{p}{)}
\end{Verbatim}


    \subsubsection{Inferir o esquema}\label{inferir-o-esquema}

    Com um \texttt{SQLContext}, estamos prontos para criar um
\texttt{DataFrame} a partir do nosso RDD existente. Mas primeiro
precisamos informar ao Spark SQL o esquema em nossos dados.

    O Spark SQL pode converter um RDD de objetos \texttt{Row} em
um\texttt{DataFrame}. As linhas são construídas passando uma lista de
pares chave/valor como \emph{kwargs} para a classe \texttt{Row}. As
chaves definem os nomes das colunas e os tipos são inferidos observando
a primeira linha. Portanto, é importante que não haja dados ausentes na
primeira linha do RDD para inferir corretamente o esquema.

    No nosso caso, primeiro precisamos dividir os dados separados por
vírgula e, em seguida, usar as informações na descrição da tarefa do KDD
de 1999 para obter os
\href{http://kdd.ics.uci.edu/databases/kddcup99/kddcup.names}{column
names}.

    \begin{Verbatim}[commandchars=\\\{\}]
{\color{incolor}In [{\color{incolor}16}]:} \PY{k+kn}{from} \PY{n+nn}{pyspark}\PY{n+nn}{.}\PY{n+nn}{sql} \PY{k}{import} \PY{n}{Row}
         
         \PY{n}{csv\PYZus{}data} \PY{o}{=} \PY{n}{raw\PYZus{}data}\PY{o}{.}\PY{n}{map}\PY{p}{(}\PY{k}{lambda} \PY{n}{l}\PY{p}{:} \PY{n}{l}\PY{o}{.}\PY{n}{split}\PY{p}{(}\PY{l+s+s2}{\PYZdq{}}\PY{l+s+s2}{,}\PY{l+s+s2}{\PYZdq{}}\PY{p}{)}\PY{p}{)}
         \PY{n}{row\PYZus{}data} \PY{o}{=} \PY{n}{csv\PYZus{}data}\PY{o}{.}\PY{n}{map}\PY{p}{(}\PY{k}{lambda} \PY{n}{p}\PY{p}{:} \PY{n}{Row}\PY{p}{(}
             \PY{n}{duration}\PY{o}{=}\PY{n+nb}{int}\PY{p}{(}\PY{n}{p}\PY{p}{[}\PY{l+m+mi}{0}\PY{p}{]}\PY{p}{)}\PY{p}{,} 
             \PY{n}{protocol\PYZus{}type}\PY{o}{=}\PY{n}{p}\PY{p}{[}\PY{l+m+mi}{1}\PY{p}{]}\PY{p}{,}
             \PY{n}{service}\PY{o}{=}\PY{n}{p}\PY{p}{[}\PY{l+m+mi}{2}\PY{p}{]}\PY{p}{,}
             \PY{n}{flag}\PY{o}{=}\PY{n}{p}\PY{p}{[}\PY{l+m+mi}{3}\PY{p}{]}\PY{p}{,}
             \PY{n}{src\PYZus{}bytes}\PY{o}{=}\PY{n+nb}{int}\PY{p}{(}\PY{n}{p}\PY{p}{[}\PY{l+m+mi}{4}\PY{p}{]}\PY{p}{)}\PY{p}{,}
             \PY{n}{dst\PYZus{}bytes}\PY{o}{=}\PY{n+nb}{int}\PY{p}{(}\PY{n}{p}\PY{p}{[}\PY{l+m+mi}{5}\PY{p}{]}\PY{p}{)}
             \PY{p}{)}
         \PY{p}{)}
\end{Verbatim}


    Uma vez que tenhamos o nosso RDD de \texttt{Row}, podemos inferir e
registrar o esquema.

    \begin{Verbatim}[commandchars=\\\{\}]
{\color{incolor}In [{\color{incolor}17}]:} \PY{n}{interactions\PYZus{}df} \PY{o}{=} \PY{n}{sqlContext}\PY{o}{.}\PY{n}{createDataFrame}\PY{p}{(}\PY{n}{row\PYZus{}data}\PY{p}{)}
         \PY{n}{interactions\PYZus{}df}\PY{o}{.}\PY{n}{registerTempTable}\PY{p}{(}\PY{l+s+s2}{\PYZdq{}}\PY{l+s+s2}{interactions}\PY{l+s+s2}{\PYZdq{}}\PY{p}{)}
\end{Verbatim}


    Agora podemos executar consultas SQL sobre nosso quadro de dados que foi
registrado como uma tabela.

    \begin{Verbatim}[commandchars=\\\{\}]
{\color{incolor}In [{\color{incolor}18}]:} \PY{c+c1}{\PYZsh{} Select tcp network interactions with more than 1 second duration and no transfer from destination}
         \PY{n}{tcp\PYZus{}interactions} \PY{o}{=} \PY{n}{sqlContext}\PY{o}{.}\PY{n}{sql}\PY{p}{(}\PY{l+s+s2}{\PYZdq{}\PYZdq{}\PYZdq{}}
         \PY{l+s+s2}{    SELECT duration, dst\PYZus{}bytes FROM interactions WHERE protocol\PYZus{}type = }\PY{l+s+s2}{\PYZsq{}}\PY{l+s+s2}{tcp}\PY{l+s+s2}{\PYZsq{}}\PY{l+s+s2}{ AND duration \PYZgt{} 1000 AND dst\PYZus{}bytes = 0}
         \PY{l+s+s2}{\PYZdq{}\PYZdq{}\PYZdq{}}\PY{p}{)}
         \PY{n}{tcp\PYZus{}interactions}\PY{o}{.}\PY{n}{show}\PY{p}{(}\PY{p}{)}
\end{Verbatim}


    \begin{Verbatim}[commandchars=\\\{\}]
+--------+---------+
|duration|dst\_bytes|
+--------+---------+
|    5057|        0|
|    5059|        0|
|    5051|        0|
|    5056|        0|
|    5051|        0|
|    5039|        0|
|    5062|        0|
|    5041|        0|
|    5056|        0|
|    5064|        0|
|    5043|        0|
|    5061|        0|
|    5049|        0|
|    5061|        0|
|    5048|        0|
|    5047|        0|
|    5044|        0|
|    5063|        0|
|    5068|        0|
|    5062|        0|
+--------+---------+
only showing top 20 rows


    \end{Verbatim}

    Os resultados das consultas SQL são RDDs e suportam todas as operações
RDD normais.

    \begin{Verbatim}[commandchars=\\\{\}]
{\color{incolor}In [{\color{incolor}19}]:} \PY{c+c1}{\PYZsh{} Output duration together with dst\PYZus{}bytes}
         \PY{n}{tcp\PYZus{}interactions\PYZus{}out} \PY{o}{=} \PY{n}{tcp\PYZus{}interactions}\PY{o}{.}\PY{n}{rdd}\PY{o}{.}\PY{n}{map}\PY{p}{(}\PY{k}{lambda} \PY{n}{p}\PY{p}{:} \PY{l+s+s2}{\PYZdq{}}\PY{l+s+s2}{Duration: }\PY{l+s+si}{\PYZob{}\PYZcb{}}\PY{l+s+s2}{, Dest. bytes: }\PY{l+s+si}{\PYZob{}\PYZcb{}}\PY{l+s+s2}{\PYZdq{}}\PY{o}{.}\PY{n}{format}\PY{p}{(}\PY{n}{p}\PY{o}{.}\PY{n}{duration}\PY{p}{,} \PY{n}{p}\PY{o}{.}\PY{n}{dst\PYZus{}bytes}\PY{p}{)}\PY{p}{)}
         \PY{k}{for} \PY{n}{ti\PYZus{}out} \PY{o+ow}{in} \PY{n}{tcp\PYZus{}interactions\PYZus{}out}\PY{o}{.}\PY{n}{collect}\PY{p}{(}\PY{p}{)}\PY{p}{:}
           \PY{n+nb}{print}\PY{p}{(}\PY{n}{ti\PYZus{}out}\PY{p}{)}
\end{Verbatim}


    \begin{Verbatim}[commandchars=\\\{\}]
Duration: 5057, Dest. bytes: 0
Duration: 5059, Dest. bytes: 0
Duration: 5051, Dest. bytes: 0
Duration: 5056, Dest. bytes: 0
Duration: 5051, Dest. bytes: 0
Duration: 5039, Dest. bytes: 0
Duration: 5062, Dest. bytes: 0
Duration: 5041, Dest. bytes: 0
Duration: 5056, Dest. bytes: 0
Duration: 5064, Dest. bytes: 0
Duration: 5043, Dest. bytes: 0
Duration: 5061, Dest. bytes: 0
Duration: 5049, Dest. bytes: 0
Duration: 5061, Dest. bytes: 0
Duration: 5048, Dest. bytes: 0
Duration: 5047, Dest. bytes: 0
Duration: 5044, Dest. bytes: 0
Duration: 5063, Dest. bytes: 0
Duration: 5068, Dest. bytes: 0
Duration: 5062, Dest. bytes: 0
Duration: 5046, Dest. bytes: 0
Duration: 5052, Dest. bytes: 0
Duration: 5044, Dest. bytes: 0
Duration: 5054, Dest. bytes: 0
Duration: 5039, Dest. bytes: 0
Duration: 5058, Dest. bytes: 0
Duration: 5051, Dest. bytes: 0
Duration: 5032, Dest. bytes: 0
Duration: 5063, Dest. bytes: 0
Duration: 5040, Dest. bytes: 0
Duration: 5051, Dest. bytes: 0
Duration: 5066, Dest. bytes: 0
Duration: 5044, Dest. bytes: 0
Duration: 5051, Dest. bytes: 0
Duration: 5036, Dest. bytes: 0
Duration: 5055, Dest. bytes: 0
Duration: 2426, Dest. bytes: 0
Duration: 5047, Dest. bytes: 0
Duration: 5057, Dest. bytes: 0
Duration: 5037, Dest. bytes: 0
Duration: 5057, Dest. bytes: 0
Duration: 5062, Dest. bytes: 0
Duration: 5051, Dest. bytes: 0
Duration: 5051, Dest. bytes: 0
Duration: 5053, Dest. bytes: 0
Duration: 5064, Dest. bytes: 0
Duration: 5044, Dest. bytes: 0
Duration: 5051, Dest. bytes: 0
Duration: 5033, Dest. bytes: 0
Duration: 5066, Dest. bytes: 0
Duration: 5063, Dest. bytes: 0
Duration: 5056, Dest. bytes: 0
Duration: 5042, Dest. bytes: 0
Duration: 5063, Dest. bytes: 0
Duration: 5060, Dest. bytes: 0
Duration: 5056, Dest. bytes: 0
Duration: 5049, Dest. bytes: 0
Duration: 5043, Dest. bytes: 0
Duration: 5039, Dest. bytes: 0
Duration: 5041, Dest. bytes: 0
Duration: 42448, Dest. bytes: 0
Duration: 42088, Dest. bytes: 0
Duration: 41065, Dest. bytes: 0
Duration: 40929, Dest. bytes: 0
Duration: 40806, Dest. bytes: 0
Duration: 40682, Dest. bytes: 0
Duration: 40571, Dest. bytes: 0
Duration: 40448, Dest. bytes: 0
Duration: 40339, Dest. bytes: 0
Duration: 40232, Dest. bytes: 0
Duration: 40121, Dest. bytes: 0
Duration: 36783, Dest. bytes: 0
Duration: 36674, Dest. bytes: 0
Duration: 36570, Dest. bytes: 0
Duration: 36467, Dest. bytes: 0
Duration: 36323, Dest. bytes: 0
Duration: 36204, Dest. bytes: 0
Duration: 32038, Dest. bytes: 0
Duration: 31925, Dest. bytes: 0
Duration: 31809, Dest. bytes: 0
Duration: 31709, Dest. bytes: 0
Duration: 31601, Dest. bytes: 0
Duration: 31501, Dest. bytes: 0
Duration: 31401, Dest. bytes: 0
Duration: 31301, Dest. bytes: 0
Duration: 31194, Dest. bytes: 0
Duration: 31061, Dest. bytes: 0
Duration: 30935, Dest. bytes: 0
Duration: 30835, Dest. bytes: 0
Duration: 30735, Dest. bytes: 0
Duration: 30619, Dest. bytes: 0
Duration: 30518, Dest. bytes: 0
Duration: 30418, Dest. bytes: 0
Duration: 30317, Dest. bytes: 0
Duration: 30217, Dest. bytes: 0
Duration: 30077, Dest. bytes: 0
Duration: 25420, Dest. bytes: 0
Duration: 22921, Dest. bytes: 0
Duration: 22821, Dest. bytes: 0
Duration: 22721, Dest. bytes: 0
Duration: 22616, Dest. bytes: 0
Duration: 22516, Dest. bytes: 0
Duration: 22416, Dest. bytes: 0
Duration: 22316, Dest. bytes: 0
Duration: 22216, Dest. bytes: 0
Duration: 21987, Dest. bytes: 0
Duration: 21887, Dest. bytes: 0
Duration: 21767, Dest. bytes: 0
Duration: 21661, Dest. bytes: 0
Duration: 21561, Dest. bytes: 0
Duration: 21455, Dest. bytes: 0
Duration: 21334, Dest. bytes: 0
Duration: 21223, Dest. bytes: 0
Duration: 21123, Dest. bytes: 0
Duration: 20983, Dest. bytes: 0
Duration: 14682, Dest. bytes: 0
Duration: 14420, Dest. bytes: 0
Duration: 14319, Dest. bytes: 0
Duration: 14198, Dest. bytes: 0
Duration: 14098, Dest. bytes: 0
Duration: 13998, Dest. bytes: 0
Duration: 13898, Dest. bytes: 0
Duration: 13796, Dest. bytes: 0
Duration: 13678, Dest. bytes: 0
Duration: 13578, Dest. bytes: 0
Duration: 13448, Dest. bytes: 0
Duration: 13348, Dest. bytes: 0
Duration: 13241, Dest. bytes: 0
Duration: 13141, Dest. bytes: 0
Duration: 13033, Dest. bytes: 0
Duration: 12933, Dest. bytes: 0
Duration: 12833, Dest. bytes: 0
Duration: 12733, Dest. bytes: 0
Duration: 12001, Dest. bytes: 0
Duration: 5678, Dest. bytes: 0
Duration: 5010, Dest. bytes: 0
Duration: 1298, Dest. bytes: 0
Duration: 1031, Dest. bytes: 0
Duration: 36438, Dest. bytes: 0

    \end{Verbatim}

    Podemos facilmente dar uma olhada no nosso esquema de quadros de dados
usando o \texttt{printSchema}.

    \begin{Verbatim}[commandchars=\\\{\}]
{\color{incolor}In [{\color{incolor}20}]:} \PY{n}{interactions\PYZus{}df}\PY{o}{.}\PY{n}{printSchema}\PY{p}{(}\PY{p}{)}
\end{Verbatim}


    \begin{Verbatim}[commandchars=\\\{\}]
root
 |-- dst\_bytes: long (nullable = true)
 |-- duration: long (nullable = true)
 |-- flag: string (nullable = true)
 |-- protocol\_type: string (nullable = true)
 |-- service: string (nullable = true)
 |-- src\_bytes: long (nullable = true)


    \end{Verbatim}

    \subsection{\texorpdfstring{Consultas como operações
\texttt{DataFrame}}{Consultas como operações DataFrame}}\label{consultas-como-operauxe7uxf5es-dataframe}

    O Spark \texttt{DataFrame} fornece uma linguagem específica de domínio
para manipulação de dados estruturados. Essa linguagem inclui métodos
que podemos concatenar para fazer seleção, filtragem, agrupamento, etc.
Por exemplo, digamos que queremos contar quantas interações existem para
cada tipo de protocolo. Nós podemos proceder da seguinte forma.

    \begin{Verbatim}[commandchars=\\\{\}]
{\color{incolor}In [{\color{incolor}21}]:} \PY{k+kn}{from} \PY{n+nn}{time} \PY{k}{import} \PY{n}{time}
         
         \PY{n}{t0} \PY{o}{=} \PY{n}{time}\PY{p}{(}\PY{p}{)}
         \PY{n}{interactions\PYZus{}df}\PY{o}{.}\PY{n}{select}\PY{p}{(}\PY{l+s+s2}{\PYZdq{}}\PY{l+s+s2}{protocol\PYZus{}type}\PY{l+s+s2}{\PYZdq{}}\PY{p}{,} \PY{l+s+s2}{\PYZdq{}}\PY{l+s+s2}{duration}\PY{l+s+s2}{\PYZdq{}}\PY{p}{,} \PY{l+s+s2}{\PYZdq{}}\PY{l+s+s2}{dst\PYZus{}bytes}\PY{l+s+s2}{\PYZdq{}}\PY{p}{)}\PY{o}{.}\PY{n}{groupBy}\PY{p}{(}\PY{l+s+s2}{\PYZdq{}}\PY{l+s+s2}{protocol\PYZus{}type}\PY{l+s+s2}{\PYZdq{}}\PY{p}{)}\PY{o}{.}\PY{n}{count}\PY{p}{(}\PY{p}{)}\PY{o}{.}\PY{n}{show}\PY{p}{(}\PY{p}{)}
         \PY{n}{tt} \PY{o}{=} \PY{n}{time}\PY{p}{(}\PY{p}{)} \PY{o}{\PYZhy{}} \PY{n}{t0}
         
         \PY{n+nb}{print}\PY{p}{(}\PY{l+s+s2}{\PYZdq{}}\PY{l+s+s2}{Query performed in }\PY{l+s+si}{\PYZob{}\PYZcb{}}\PY{l+s+s2}{ seconds}\PY{l+s+s2}{\PYZdq{}}\PY{o}{.}\PY{n}{format}\PY{p}{(}\PY{n+nb}{round}\PY{p}{(}\PY{n}{tt}\PY{p}{,}\PY{l+m+mi}{3}\PY{p}{)}\PY{p}{)}\PY{p}{)}
\end{Verbatim}


    \begin{Verbatim}[commandchars=\\\{\}]
+-------------+------+
|protocol\_type| count|
+-------------+------+
|          tcp|190065|
|          udp| 20354|
|         icmp|283602|
+-------------+------+

Query performed in 10.102 seconds

    \end{Verbatim}

    Agora imagine que queremos contar quantas interações duram mais de 1
segundo, sem transferência de dados do destino, agrupadas por tipo de
protocolo. Podemos apenas adicionar para filtrar chamadas para o
anterior.

    \begin{Verbatim}[commandchars=\\\{\}]
{\color{incolor}In [{\color{incolor}22}]:} \PY{n}{t0} \PY{o}{=} \PY{n}{time}\PY{p}{(}\PY{p}{)}
         \PY{n}{interactions\PYZus{}df}\PY{o}{.}\PY{n}{select}\PY{p}{(}\PY{l+s+s2}{\PYZdq{}}\PY{l+s+s2}{protocol\PYZus{}type}\PY{l+s+s2}{\PYZdq{}}\PY{p}{,} \PY{l+s+s2}{\PYZdq{}}\PY{l+s+s2}{duration}\PY{l+s+s2}{\PYZdq{}}\PY{p}{,} \PY{l+s+s2}{\PYZdq{}}\PY{l+s+s2}{dst\PYZus{}bytes}\PY{l+s+s2}{\PYZdq{}}\PY{p}{)}\PY{o}{.}\PY{n}{filter}\PY{p}{(}\PY{n}{interactions\PYZus{}df}\PY{o}{.}\PY{n}{duration}\PY{o}{\PYZgt{}}\PY{l+m+mi}{1000}\PY{p}{)}\PY{o}{.}\PY{n}{filter}\PY{p}{(}\PY{n}{interactions\PYZus{}df}\PY{o}{.}\PY{n}{dst\PYZus{}bytes}\PY{o}{==}\PY{l+m+mi}{0}\PY{p}{)}\PY{o}{.}\PY{n}{groupBy}\PY{p}{(}\PY{l+s+s2}{\PYZdq{}}\PY{l+s+s2}{protocol\PYZus{}type}\PY{l+s+s2}{\PYZdq{}}\PY{p}{)}\PY{o}{.}\PY{n}{count}\PY{p}{(}\PY{p}{)}\PY{o}{.}\PY{n}{show}\PY{p}{(}\PY{p}{)}
         \PY{n}{tt} \PY{o}{=} \PY{n}{time}\PY{p}{(}\PY{p}{)} \PY{o}{\PYZhy{}} \PY{n}{t0}
         
         \PY{n+nb}{print}\PY{p}{(}\PY{l+s+s2}{\PYZdq{}}\PY{l+s+s2}{Query performed in }\PY{l+s+si}{\PYZob{}\PYZcb{}}\PY{l+s+s2}{ seconds}\PY{l+s+s2}{\PYZdq{}}\PY{o}{.}\PY{n}{format}\PY{p}{(}\PY{n+nb}{round}\PY{p}{(}\PY{n}{tt}\PY{p}{,}\PY{l+m+mi}{3}\PY{p}{)}\PY{p}{)}\PY{p}{)}
\end{Verbatim}


    \begin{Verbatim}[commandchars=\\\{\}]
+-------------+-----+
|protocol\_type|count|
+-------------+-----+
|          tcp|  139|
+-------------+-----+

Query performed in 10.741 seconds

    \end{Verbatim}

    Podemos usar isso para realizar algumas
\href{http://en.wikipedia.org/wiki/Exploratory_data_analysis}{exploratory
data analysis}. Vamos contar quantas interações normais e de ataque nós
temos. Primeiro, precisamos adicionar a coluna de rótulo aos nossos
dados.

    \begin{Verbatim}[commandchars=\\\{\}]
{\color{incolor}In [{\color{incolor}23}]:} \PY{k}{def} \PY{n+nf}{get\PYZus{}label\PYZus{}type}\PY{p}{(}\PY{n}{label}\PY{p}{)}\PY{p}{:}
             \PY{k}{if} \PY{n}{label}\PY{o}{!=}\PY{l+s+s2}{\PYZdq{}}\PY{l+s+s2}{normal.}\PY{l+s+s2}{\PYZdq{}}\PY{p}{:}
                 \PY{k}{return} \PY{l+s+s2}{\PYZdq{}}\PY{l+s+s2}{attack}\PY{l+s+s2}{\PYZdq{}}
             \PY{k}{else}\PY{p}{:}
                 \PY{k}{return} \PY{l+s+s2}{\PYZdq{}}\PY{l+s+s2}{normal}\PY{l+s+s2}{\PYZdq{}}
             
         \PY{n}{row\PYZus{}labeled\PYZus{}data} \PY{o}{=} \PY{n}{csv\PYZus{}data}\PY{o}{.}\PY{n}{map}\PY{p}{(}\PY{k}{lambda} \PY{n}{p}\PY{p}{:} \PY{n}{Row}\PY{p}{(}
             \PY{n}{duration}\PY{o}{=}\PY{n+nb}{int}\PY{p}{(}\PY{n}{p}\PY{p}{[}\PY{l+m+mi}{0}\PY{p}{]}\PY{p}{)}\PY{p}{,} 
             \PY{n}{protocol\PYZus{}type}\PY{o}{=}\PY{n}{p}\PY{p}{[}\PY{l+m+mi}{1}\PY{p}{]}\PY{p}{,}
             \PY{n}{service}\PY{o}{=}\PY{n}{p}\PY{p}{[}\PY{l+m+mi}{2}\PY{p}{]}\PY{p}{,}
             \PY{n}{flag}\PY{o}{=}\PY{n}{p}\PY{p}{[}\PY{l+m+mi}{3}\PY{p}{]}\PY{p}{,}
             \PY{n}{src\PYZus{}bytes}\PY{o}{=}\PY{n+nb}{int}\PY{p}{(}\PY{n}{p}\PY{p}{[}\PY{l+m+mi}{4}\PY{p}{]}\PY{p}{)}\PY{p}{,}
             \PY{n}{dst\PYZus{}bytes}\PY{o}{=}\PY{n+nb}{int}\PY{p}{(}\PY{n}{p}\PY{p}{[}\PY{l+m+mi}{5}\PY{p}{]}\PY{p}{)}\PY{p}{,}
             \PY{n}{label}\PY{o}{=}\PY{n}{get\PYZus{}label\PYZus{}type}\PY{p}{(}\PY{n}{p}\PY{p}{[}\PY{l+m+mi}{41}\PY{p}{]}\PY{p}{)}
             \PY{p}{)}
         \PY{p}{)}
         \PY{n}{interactions\PYZus{}labeled\PYZus{}df} \PY{o}{=} \PY{n}{sqlContext}\PY{o}{.}\PY{n}{createDataFrame}\PY{p}{(}\PY{n}{row\PYZus{}labeled\PYZus{}data}\PY{p}{)}
\end{Verbatim}


    Desta vez, não precisamos registrar o esquema, pois vamos usar a
interface de consulta OO.

    Vamos verificar o anterior realmente funciona contando dados de ataque e
normal em nosso quadro de dados.

    \begin{Verbatim}[commandchars=\\\{\}]
{\color{incolor}In [{\color{incolor}24}]:} \PY{n}{t0} \PY{o}{=} \PY{n}{time}\PY{p}{(}\PY{p}{)}
         \PY{n}{interactions\PYZus{}labeled\PYZus{}df}\PY{o}{.}\PY{n}{select}\PY{p}{(}\PY{l+s+s2}{\PYZdq{}}\PY{l+s+s2}{label}\PY{l+s+s2}{\PYZdq{}}\PY{p}{)}\PY{o}{.}\PY{n}{groupBy}\PY{p}{(}\PY{l+s+s2}{\PYZdq{}}\PY{l+s+s2}{label}\PY{l+s+s2}{\PYZdq{}}\PY{p}{)}\PY{o}{.}\PY{n}{count}\PY{p}{(}\PY{p}{)}\PY{o}{.}\PY{n}{show}\PY{p}{(}\PY{p}{)}
         \PY{n}{tt} \PY{o}{=} \PY{n}{time}\PY{p}{(}\PY{p}{)} \PY{o}{\PYZhy{}} \PY{n}{t0}
         
         \PY{n+nb}{print}\PY{p}{(}\PY{l+s+s2}{\PYZdq{}}\PY{l+s+s2}{Query performed in }\PY{l+s+si}{\PYZob{}\PYZcb{}}\PY{l+s+s2}{ seconds}\PY{l+s+s2}{\PYZdq{}}\PY{o}{.}\PY{n}{format}\PY{p}{(}\PY{n+nb}{round}\PY{p}{(}\PY{n}{tt}\PY{p}{,}\PY{l+m+mi}{3}\PY{p}{)}\PY{p}{)}\PY{p}{)}
\end{Verbatim}


    \begin{Verbatim}[commandchars=\\\{\}]
+------+------+
| label| count|
+------+------+
|normal| 97278|
|attack|396743|
+------+------+

Query performed in 10.21 seconds

    \end{Verbatim}

    Agora queremos contá-las por rótulo e tipo de protocolo, para ver o quão
importante é o tipo de protocolo detectar quando uma interação é ou não
um ataque.

    \begin{Verbatim}[commandchars=\\\{\}]
{\color{incolor}In [{\color{incolor}25}]:} \PY{n}{t0} \PY{o}{=} \PY{n}{time}\PY{p}{(}\PY{p}{)}
         \PY{n}{interactions\PYZus{}labeled\PYZus{}df}\PY{o}{.}\PY{n}{select}\PY{p}{(}\PY{l+s+s2}{\PYZdq{}}\PY{l+s+s2}{label}\PY{l+s+s2}{\PYZdq{}}\PY{p}{,} \PY{l+s+s2}{\PYZdq{}}\PY{l+s+s2}{protocol\PYZus{}type}\PY{l+s+s2}{\PYZdq{}}\PY{p}{)}\PY{o}{.}\PY{n}{groupBy}\PY{p}{(}\PY{l+s+s2}{\PYZdq{}}\PY{l+s+s2}{label}\PY{l+s+s2}{\PYZdq{}}\PY{p}{,} \PY{l+s+s2}{\PYZdq{}}\PY{l+s+s2}{protocol\PYZus{}type}\PY{l+s+s2}{\PYZdq{}}\PY{p}{)}\PY{o}{.}\PY{n}{count}\PY{p}{(}\PY{p}{)}\PY{o}{.}\PY{n}{show}\PY{p}{(}\PY{p}{)}
         \PY{n}{tt} \PY{o}{=} \PY{n}{time}\PY{p}{(}\PY{p}{)} \PY{o}{\PYZhy{}} \PY{n}{t0}
         
         \PY{n+nb}{print}\PY{p}{(}\PY{l+s+s2}{\PYZdq{}}\PY{l+s+s2}{Query performed in }\PY{l+s+si}{\PYZob{}\PYZcb{}}\PY{l+s+s2}{ seconds}\PY{l+s+s2}{\PYZdq{}}\PY{o}{.}\PY{n}{format}\PY{p}{(}\PY{n+nb}{round}\PY{p}{(}\PY{n}{tt}\PY{p}{,}\PY{l+m+mi}{3}\PY{p}{)}\PY{p}{)}\PY{p}{)}
\end{Verbatim}


    \begin{Verbatim}[commandchars=\\\{\}]
+------+-------------+------+
| label|protocol\_type| count|
+------+-------------+------+
|normal|          udp| 19177|
|normal|         icmp|  1288|
|normal|          tcp| 76813|
|attack|         icmp|282314|
|attack|          tcp|113252|
|attack|          udp|  1177|
+------+-------------+------+

Query performed in 9.532 seconds

    \end{Verbatim}

    À primeira vista, parece que as interações \emph{udp} estão em menor
proporção entre ataques de rede e outros tipos de protocolos.

    E podemos fazer agrupamentos muito mais sofisticados. Por exemplo,
adicione ao anterior um "split" baseado na transferência de dados do
destino.

    \begin{Verbatim}[commandchars=\\\{\}]
{\color{incolor}In [{\color{incolor}26}]:} \PY{n}{t0} \PY{o}{=} \PY{n}{time}\PY{p}{(}\PY{p}{)}
         \PY{n}{interactions\PYZus{}labeled\PYZus{}df}\PY{o}{.}\PY{n}{select}\PY{p}{(}\PY{l+s+s2}{\PYZdq{}}\PY{l+s+s2}{label}\PY{l+s+s2}{\PYZdq{}}\PY{p}{,} \PY{l+s+s2}{\PYZdq{}}\PY{l+s+s2}{protocol\PYZus{}type}\PY{l+s+s2}{\PYZdq{}}\PY{p}{,} \PY{l+s+s2}{\PYZdq{}}\PY{l+s+s2}{dst\PYZus{}bytes}\PY{l+s+s2}{\PYZdq{}}\PY{p}{)}\PY{o}{.}\PY{n}{groupBy}\PY{p}{(}\PY{l+s+s2}{\PYZdq{}}\PY{l+s+s2}{label}\PY{l+s+s2}{\PYZdq{}}\PY{p}{,} \PY{l+s+s2}{\PYZdq{}}\PY{l+s+s2}{protocol\PYZus{}type}\PY{l+s+s2}{\PYZdq{}}\PY{p}{,} \PY{n}{interactions\PYZus{}labeled\PYZus{}df}\PY{o}{.}\PY{n}{dst\PYZus{}bytes}\PY{o}{==}\PY{l+m+mi}{0}\PY{p}{)}\PY{o}{.}\PY{n}{count}\PY{p}{(}\PY{p}{)}\PY{o}{.}\PY{n}{show}\PY{p}{(}\PY{p}{)}
         \PY{n}{tt} \PY{o}{=} \PY{n}{time}\PY{p}{(}\PY{p}{)} \PY{o}{\PYZhy{}} \PY{n}{t0}
         
         \PY{n+nb}{print}\PY{p}{(}\PY{l+s+s2}{\PYZdq{}}\PY{l+s+s2}{Query performed in }\PY{l+s+si}{\PYZob{}\PYZcb{}}\PY{l+s+s2}{ seconds}\PY{l+s+s2}{\PYZdq{}}\PY{o}{.}\PY{n}{format}\PY{p}{(}\PY{n+nb}{round}\PY{p}{(}\PY{n}{tt}\PY{p}{,}\PY{l+m+mi}{3}\PY{p}{)}\PY{p}{)}\PY{p}{)}
\end{Verbatim}


    \begin{Verbatim}[commandchars=\\\{\}]
+------+-------------+---------------+------+
| label|protocol\_type|(dst\_bytes = 0)| count|
+------+-------------+---------------+------+
|normal|          udp|          false| 15583|
|attack|          udp|          false|    11|
|attack|          tcp|           true|110583|
|normal|          tcp|          false| 67500|
|attack|         icmp|           true|282314|
|attack|          tcp|          false|  2669|
|normal|          tcp|           true|  9313|
|normal|          udp|           true|  3594|
|normal|         icmp|           true|  1288|
|attack|          udp|           true|  1166|
+------+-------------+---------------+------+

Query performed in 10.407 seconds

    \end{Verbatim}

    Vemos quão relevante é essa nova divisão para determinar se uma
interação de rede é um ataque.

    Vamos parar por aqui, mas podemos ver como esse tipo de consulta é
poderosa para explorar nossos dados. Na verdade, podemos replicar todas
as divisões que vimos nos cadernos anteriores, ao introduzir árvores de
classificação, apenas selecionando, tateando e filtrando nosso
dataframe. Para uma lista mais detalhada (mas menos real) das operações
e fontes de dados do DataFrame do Spark, dê uma olhada na documentação
oficial
\href{https://spark.apache.org/docs/latest/sql-programming-guide.html\#dataframe-operations}{aqui}.


    % Add a bibliography block to the postdoc
    
    
    
    \end{document}
